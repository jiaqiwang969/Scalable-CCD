\documentclass[11pt]{ctexart}
\usepackage{geometry}
\geometry{a4paper,margin=2.2cm}
\usepackage{booktabs}
\usepackage{multirow}
\usepackage{siunitx}
\usepackage{longtable}
\usepackage{hyperref}
\hypersetup{colorlinks=true,linkcolor=blue,urlcolor=magenta}
\title{Scalable CCD CUDA 与 Metal 验证总结}
\author{}
\date{\today}
\begin{document}
\maketitle

\section{实验环境}
\begin{itemize}
    \item 操作系统:Ubuntu 24.04.2 LTS,内核与用户态均为 64-bit。
    \item CPU:Intel Core Ultra 9 185H(22 逻辑核),用于 CPU 宽阶段参考测试。
    \item GPU:\texttt{GPU0} 为 NVIDIA RTX 2000 Ada Generation Laptop(8~GB),\texttt{GPU1} 为 NVIDIA GeForce RTX~3090(24~GB),驱动 580.95.05,CUDA 12.6.0 运行时。
    \item 编译:`cmake -S . -B build -DCMAKE_BUILD_TYPE=Release -DSCALABLE\_CCD\_WITH\_CUDA=ON -DSCALABLE\_CCD\_BUILD\_TESTS=ON`,随后 `cmake --build build -j`。
    \item 验证:使用 Catch2 测试 `scalable\_ccd\_tests`,通过 `--durations yes` 获取壁钟时间;通过 `CUDA\_VISIBLE\_DEVICES` 分别固定至 RTX 2000 Ada 与 RTX~3090。
    \item Metal 结果:来自 `../Scalable-CCD-02/tests/results/metal_sap_cloth_ball_{vf,ee}.json`,由 macOS + Metal 实测提供。
\end{itemize}

\section{测试方法}
\begin{enumerate}
    \item 宽阶段 GPU:对 `tests/test_broad_phase.cu` 中的 5 个 Section(Armadillo-Rollers、Cloth-Ball、Cloth-Funnel、N-Body、Rod-Twist)分别运行 `./tests/scalable_ccd_tests "Test CUDA broad phase" -c <Section> --durations yes`,记录 Catch2 输出的单节壁钟时间。
    \item 宽阶段 CPU:运行 `./tests/scalable_ccd_tests "Test CPU broad phase" --durations yes`,得到 Cloth-Ball 参考值。
    \item 窄阶段 GPU:运行 `./tests/scalable_ccd_tests "Test CUDA narrow phase" --durations yes`,测量 Cloth-Ball 时间。
    \item Metal:直接引用 JSON 中的 `gpu_ms`、`cpu_ms`、`cpu_total_ms` 字段,分别对应 SAP 内核、Host 侧调度与包含读写的总耗时。
\end{enumerate}
所有测试均使用 `SCALABLE\_CCD\_USE\_DOUBLE` 精度并对比 ground-truth,保证功能正确。

\section{宽阶段性能对比}
表~\ref{tab:broadphase} 汇总了不同 GPU 与 CPU 的 Catch2 壁钟时间(单位:秒)。CPU 行为为纯 CPU Sort-and-Sweep;GPU 数据包含 mesh 解析、AABB 构建、device 复制与两次 `BroadPhase::detect_overlaps()`。

\begin{table}[htbp]
    \centering
    \caption{宽阶段各场景壁钟时间(秒)}
    \label{tab:broadphase}
    \begin{tabular}{lccc}
        \toprule
        场景 & RTX 2000 Ada & RTX 3090 & CPU (Sort-and-Sweep) \\
        \midrule
        Armadillo-Rollers & 7.870 & 8.993 & \multicolumn{1}{c}{---} \\
        Cloth-Ball        & 5.727 & 4.863 & 3.091 \\
        Cloth-Funnel      & 0.309 & 0.330 & \multicolumn{1}{c}{---} \\
        N-Body            & 23.031 & 17.848 & \multicolumn{1}{c}{---} \\
        Rod-Twist         & 1.544 & 1.393 & \multicolumn{1}{c}{---} \\
        \bottomrule
    \end{tabular}
\end{table}

可以看到:
\begin{itemize}
    \item RTX~3090 在大多数场景上快于 RTX 2000 Ada,尤其是粒子数更多的 N-Body(缩短约 22\%)。
    \item Armadillo-Rollers 场景 RTX~3090 反而略慢(8.99~s vs 7.87~s),原因可能是 3090 进入 P8 低功耗状态,需要额外预热时间。
    \item CPU 宽阶段(Cloth-Ball)只需 3.09~s,反映出当前 GPU 路径仍将大量时间花在 host 前处理与 device 拷贝上,尚未展现潜在加速比。
\end{itemize}

\section{窄阶段性能}
窄阶段仅在 Cloth-Ball 数据上验证,结果如表~\ref{tab:narrow}. RTX 2000 Ada 更快(0.653~s),RTX~3090 因实时功耗/调度影响慢约 24\%。

\begin{table}[htbp]
    \centering
    \caption{Cloth-Ball 窄阶段壁钟时间(秒)}
    \label{tab:narrow}
    \begin{tabular}{lcc}
        \toprule
        后端 & RTX 2000 Ada & RTX 3090 \\
        \midrule
        CUDA Narrow Phase & 0.653 & 0.807 \\
        \bottomrule
    \end{tabular}
\end{table}

\section{Cloth-Ball 多后端(Metal/CPU/CUDA)}
表~\ref{tab:clothball} 对比了 Cloth-Ball 的不同实现。Metal 数据来自 JSON,单位统一为毫秒。

\begin{table}[htbp]
    \centering
    \caption{Cloth-Ball 宽阶段跨后端耗时对比(毫秒)}
    \label{tab:clothball}
    \begin{tabular}{llccc}
        \toprule
        后端 & 数据分量 & Host 核心 & Host 总计 & GPU 内核或壁钟 \\
        \midrule
        Metal & VF SAP & 10.792 & 65.982 & 9.632 \\
        Metal & EE SAP & 13.114 & 30.218 & 8.123 \\
        CPU (Ultra 9) & VF+EE & \multicolumn{1}{c}{---} & 3\,091 & 3\,091 \\
        CUDA RTX 2000 Ada & VF+EE & \multicolumn{1}{c}{---} & 5\,727 & 5\,727 \\
        CUDA RTX 3090 & VF+EE & \multicolumn{1}{c}{---} & 4\,863 & 4\,863 \\
        \bottomrule
    \end{tabular}
\end{table}

说明:
\begin{itemize}
    \item Metal 报告的是“内核 + host glue”级别的毫秒级耗时,且将 VF 与 EE 拆分;
    \item 本次 CUDA/CPU 数据是单次 Catch2 Section 的壁钟时间,包含 mesh 读取、AABB 构建与 VF/EE 两次检测,因而显著大于 Metal 的核级时间;
    \item 即便在 RTX~3090 上,当前实现仍远慢于金属版-vf/ee 核心(4.86~s vs 0.018~s),表明主要瓶颈在数据搬运与 kernel 配置开销。
\end{itemize}

\section{差异分析与结论}
\begin{enumerate}
    \item \textbf{硬件差异:}RTX~3090 拥有 10496 CUDA Core 与 936~GB/s 带宽,理论上显著强于 RTX~2000 Ada(3072 Core,256-bit LPDDR5)。在重算例(N-Body、Rod-Twist)可见 15--25\% 的加速;而在 Armadillo 等轻载任务中,两者差距被调度噪声掩盖。
    \item \textbf{CPU 与 GPU 路径:}CPU Sort-and-Sweep 在 Cloth-Ball 上 3.09~s,优于当前 CUDA 路径,说明 GPU 管线还需减少 host 预处理(如 mesh 解析复用)并延长 kernel 批处理时间才值得搬上 GPU。
    \item \textbf{Metal 对比:}Metal 版本将 SAP 拆成 VF/EE 并直接在 GPU 核心中维护工作队列,真正的 GPU 计算只需约 10~ms;而我们的 Linux/CUDA 测试主要花在初始化、内存调度和跨主机的数据整理上。若要缩小差距,需要开启 `SCALABLE\_CCD\_WITH\_PROFILER` 并在 CUDA 核心内部采集 `GPUProfilePoint`,结合持久化的 device 缓冲(减少 `BroadPhase::build` 重复分配)。
    \item \textbf{下一步:}建议将 Cloth-Ball 以外的数据集也制作 Metal 结果,或在 CUDA 侧输出 JSON 格式的运行统计,与 `tests/results` 目录的报告脚本对齐,才能形成一键对比。
\end{enumerate}

\end{document}
